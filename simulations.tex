\section{Synthetic data analyses}


To demonstrate the package functionalities, we present the analysis of three
simulated datasets. For all the simulated datasets, we generate data across 84
patches within a region that goes from Cape Hatteras in North Carolina~(USA) to
the border between Canada and Maine~(USA), from just north of~$35^\circ$~N to
above~$44^\circ$~N. This study region has been the object of study of an analysis of
Summer flounder~(\textit{Paralichthys dentatus})
data~\citep{fredston2025dynamic}. For each of those patches, we simulate data
across 29 time-points, ranging from 1993 to 2021. We obtained sea surface
temperature~(SST), sea surface salinity~(SSS), northward sea velocity~(NSV), and
eastward sea velocity~(ESV) from the Global Ocean Physics Reanalysis product
from the European Union Copernicus Marine Service
Information~\citep{cmems2025global}. SST is measured in degrees celsius, SSS in
particle salinity units, and the velocities in meters per second. We use data
for the month of April, mimicking the scenario where yearly data are collected
always at the same time of the year. Next, we averaged the satellite-derived
variables within each patch. The panel in Figure~\ref{fig:sst} depicts the
``raw'' SST on the left and the averaged SST on the right.

\begin{figure}[tb]
  \centering
  \includegraphics{img/sst_sim.pdf}
  \caption{On the left, the raw satellite-derived SST. On the right, the same
    variable averaged across patches.}\label{fig:sst}
\end{figure}


The idea is to demonstrate how these data can be analyzed using the
\texttt{drmr} package and show that the model estimates match the
data-generating process.

\subsection{Scenario 1: Recruitment \& environment}

We simulate a dataset where the log-transformed recruitment has a quadratic
relationship with temperature. Assuming the structure in
Equation~\eqref{eq:lmrec}, we
set~$\mathbf{x}^{(r)}_{t, i} = [1 \quad \mathrm{SST}_{t, i} \quad \mathrm{SST}^2_{t,
  i}]^{\top}$, where~$\mathrm{SST}_{t, i}$ represents the~(centered and scaled) SST
at time~$t$ and patch~$i$. The vector of regression
coefficients~$\mathbf{\beta}_r$ was set
to~$[-2 \quad -0.96 \quad -0.25]^{\top}$. These values were chosen so that the temperature
at which recruitment is maximized is around $13 \mathrm{C}^{\circ}$. In addition, an
AR(1) term~(with $\alpha = 0.4$ and $\tau^2 = 0.5$) was included to the linear predictor
to introduce temporal dependence often observed in fisheries
data~\citep{johnson2016can}.


After using the aforementioned parameters to establish the relationship between
recruitment and the environment, we use the assumptions of our DRM to simulate
the observed density. \lcg{VALUES FOR $\phi$ and $\rho_{t, i}$ HERE}.



\lcg{[PLACEHOLDER: INDUCED RELATIONSHIP BETWEEN RECRUITMENT AND OBSERVED
  DENSITY]}

\subsection{Scenario 2: Survival, environment, \& fishing mortality}

We simulate a dataset where the log-transformed survival has (think about the
relationship) with temperature. Assuming the structure in
Equation~\eqref{eq:lmsurv}, we
set~$\mathbf{x}^{(s)}_{t, i} = [1 \quad \mathrm{SST}_{t, i} \quad \mathrm{SST}^2_{t,
  i}$, where $\mathrm{SST}_{t, i}$ represents the~(centered and scaled) SST at
time~$t$ and patch~$i$. The vector of regression coefficients~$\mathbf{\beta}_s$ was
set to~$[\cdots]$. These values were chosen so that the temperature at which survival
is maximized is around $12 \mathrm{C}^{\circ}$. In addition, an AR(1) term~(with
$\alpha = 0.4$ and $\tau^2 = 0.5$) was included to the recruitment linear predictor to
introduce temporal dependence, similarly to the previous Section.


\subsection{Scenario 3: Recruitment, environment, fishing mortality, \&
  movement}

\subsection{Scenario 4: Recruitment, survival, environment, fishing mortality,
  \& movement}

%%% Local Variables:
%%% mode: LaTeX
%%% TeX-master: t
%%% End:
